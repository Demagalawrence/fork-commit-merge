\documentclass{article}
\usepackage{amsmath}
\usepackage{amsfonts}

\title{Typesetting Mathematics in LaTeX}
\author{Megha B}
\date{\today}

\begin{document}
\maketitle

\section*{Fractions}
We can typeset fractions using the \texttt{\textbackslash frac} command.  
Here is an example of a fraction equation:
\[
\frac{a}{b} + \frac{c}{d} = \frac{ad + bc}{bd}
\]
This equation shows how two fractions are added by converting them to a common denominator.

\section*{Integrals}
To typeset an integral in LaTeX, we use the \texttt{\textbackslash int} command with limits.  
For example:
\[
\int_a^b f(x) \, dx
\]
Here, \( a \) and \( b \) represent the lower and upper limits of integration respectively, and \( f(x) \) is the function being integrated.

\section*{Matrices}
LaTeX allows us to typeset matrices using the \texttt{bmatrix} environment.  
Here is a 2x2 matrix:
\[
\begin{bmatrix}
a & b \\
c & d
\end{bmatrix}
\]
Each element \( a, b, c, d \) represents an entry in the matrix.  
- \( a \) is the element in the first row and first column,  
- \( b \) is in the first row and second column,  
- \( c \) is in the second row and first column, and  
- \( d \) is in the second row and second column.

\end{document}
